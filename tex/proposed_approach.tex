
\subsection{Dependency Graph}

We will name a graph that is defined by a call graph and Python class and module structure a \emph{dependency graph}.

The graph's nodes are filenames, functions and classes that are extracted using Python \textt{ast} library. In this graph we have file-function edges whenever a function is defined in a given file, and function-function edges for function calls.

Formally

$$E = E_{file} \cup E_{functions} \cup E_{calls}$$

where

\begin{gather*}
E_{file} = \{(repo, filename) | repo \texttt{ contains } filename\} \\
E_{functions} = \bigcup\limits_{filename} \{(filename, func) | filename \texttt{ defines function } func\} \\
E_{calls} = \bigcup\limits_{func} \{(func, called\_func) | func \texttt{ calls } called\_func\}
\end{gather*}

Intuitively $(v, w) \in E$ if $w$ is defined/contained in $v$, where $v$ is from one level higher (root-repo, repo-file, file-function, function-called function), Thus vertices form dependency hierarchy, where next level contains elements that dependend on elements from previous levels.
